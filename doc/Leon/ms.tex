% Every Latex document starts with a documentclass command
\documentclass[a4paper, 11pt]{article}

% Load some packages
\usepackage{graphicx} % This allows you to put figures in
\usepackage{natbib}   % This allows for relatively pain-free reference lists
\usepackage[left=3cm,top=3cm,right=3cm]{geometry} % The way I like the margins

% This helps with figure placement
\renewcommand{\topfraction}{0.85}
\renewcommand{\textfraction}{0.1}
\parindent=0cm
\newcommand{\yy}{\mathbf{y}}
\newcommand{\mm}{\mathbf{m}_\theta}
\newcommand{\CC}{\mathbf{C}_\theta}

% Set values so you can have a title
\title{}
\author{}
\date{} % Otherwise, it'll show today's date

% Document starts here
\begin{document}


Suppose that the underlying QSO variability as a function of time is
described by:
\begin{eqnarray}
y_{\rm QSO}(t)
\end{eqnarray}
Then the light curves of the two images $A$ and $B$ will be given by:
\begin{eqnarray}
y_A(t) &=& y_{\rm QSO}(t) + \mu_A(t)\\
y_B(t) &=& y_{\rm QSO}(t - \tau) + \mu_B(t)
\end{eqnarray}
where $\mu_A(t)$ and $\mu_B(t)$ are microlensing perturbations.



The data for all images are in magnitudes:
\begin{eqnarray}
\yy = \{y_1, y_2, ..., y_N\}
\end{eqnarray}


The probability distribution for the data $\yy$ given the parameters
$\theta$
is a multivariate normal distribution, where the mean vector $\mm$
and the covariance matrix $\CC$ depend on the parameters:
\begin{eqnarray}
p(\yy | \theta) &=& \frac{1}{\sqrt{(2\pi)^N\det{\CC}}}
\exp\left[
-\frac{1}{2}
\left(
\yy - \mm
\right)^T
\CC^{-1}
\left(
\yy - \mm
\right)
\right]
\end{eqnarray}

Numerically, the logarithm of the determinant and the
$\CC^{-1}\left(\yy - \mm\right)$ term are implemented using
Cholesky decompositions.

\end{document}

